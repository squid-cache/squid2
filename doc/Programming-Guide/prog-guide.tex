% $Id$
\documentstyle[11pt,path,psfig]{report}
%\parskip		1ex
%\topmargin		0pt
%\headheight		0pt
%\headsep		0pt
%\marginparwidth	0pt
%\marginparsep		0pt
%\oddsidemargin		0pt
%\evensidemargin	0pt
%\textheight		9in
%\textwidth		6.5in
%\parindent		0em

\newenvironment{SS}{\singlespace}{}

\hyphenation{time-stamps time-stamp net-work ac-ces-ses ac-ces-sed}

\begin{document}
\bibliographystyle{ieeetr}

\author{Duane Wessels\\
Squid Developers}
\title{Squid Programmers Guide}

\maketitle

\begin{abstract}
Squid is a WWW Cache application developed by the National Laboratory
for Applied Network Research and members of the Web Caching community.
Squid is implemented as a single, non-blocking process based around
a BSD select() loop.  This document describes the operation of the Squid
source code and is intended to be used by others who wish to customize
or improve it.
\end{abstract}

%%
%% Chapter : INTRODUCTION
%%
\chapter{Introduction}

The Squid source code has evolved more from empirical observation and
tinkering, rather than a solid design process.  It carries a legacy of
being ``touched'' by numerous individuals, each with somewhat different
techniques and terminology.  

Squid is a single-process proxy server.  Every request is handled by
the main process, with the exception of FTP.  However, Squid does not
use a ``threads package'' such has Pthreads.  While this might be 
easier to code, it suffers from portability and performance problems.
Instead Squid maintains data structures and state information for
each active request.

%%
%% Chapter : MAIN LOOP
%%
\chapter{The Main Loop}

At the core of Squid is the {\tt select(2)} system call.  Squid uses
{\tt select()} (or alternatively {\tt poll(2)} in recent versions) to 
process I/O on all open file descriptors.

%%
%% Chapter : DATA STRUCTURES
%%
\chapter{Data Structures}

%%
%% Chapter : STORAGE MANAGER
%%
\chapter{Storage Manager}

%%
%% Chapter : IP CACHE
%%
\chapter{IP Cache}

%%
%% Chapter : SERVER PROTOCOLS
%%
\chapter{Server Protocols}
\section{HTTP}
\section{FTP}
\section{Gopher}
\section{Wais}
\section{SSL}
\section{Passthrough}

%%
%% Chapter : ACCESS CONTROLS
%%
\chapter{Access Controls}

%%
%% Chapter : ICP
%%
\chapter{ICP}

%%
%% Chapter : CACHE MANAGER
%%
\chapter{Cache Manager}

%%%%%%
%%%%%% BIBLIOGRAPHY
%%%%%%
\newpage \bibliography{references}

% perl -ne 'printf ("\\nocite{\%s}\n", $1) if (/^@\w+{(\w+),/);' references.bib

\nocite{rfc850}
\nocite{rfc1123}

\end{document}
